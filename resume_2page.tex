\documentclass[letterpaper,hidelinks]{scrartcl}
\usepackage{nicefrac}
\usepackage{plainresume3}
\usepackage{changepage} % used for adjustwidth. Not required if those are removed
\pagestyle{plain}
%\author{Keyan Pishdadian}

\begin{document}
\title{keyan pishdadian}

(774) 273-3047 \hfill \href{https://keyanp.com}{\texttt{keyanp.com}} \\
\href{mailto:kpishdadian@gmail.com}{\texttt{kpishdadian@gmail.com}} \hfill \href{https://github.com/keyan}{\texttt{github.com/keyan}} \\
\null \hfill \href{https://linkedin.com/in/keyanp}{\texttt{linkedin.com/in/keyanp}}

%
%-------------------------------------------------------------------------------
%

\mag 1000

%
%-------------------------------------------------------------------------------
%

\section*{Experience}

\begin{list1}

\item \begin{tabular1bold} Lyft, Senior Software Engineer; Seattle, WA & March 2019 -- Present \end{tabular1bold}
\item \begin{tabular1bold} Lyft, Software Engineer; Seattle, WA & March 2018 -- March 2019 \end{tabular1bold}

  \begin{list2}
  \item Working on a small and diverse team developing alternative (to cars) transportation options and multimodal journey planning. Went from concepts to shipping transit journey planning, appearing alongside car options in several large metro areas.
  \item Developed novel routing engine which can compute multimodal journeys in transit dense regions in $<$100ms. Designed components to manage transportation network data structures and realtime updates with concurrent access. \hfill\emph{Go, C++}
  \item Productionized research using fuzzy logic for multicriteria evaluation of transit journeys. \hfill\emph{C++}
  \item Designed, developed, and shipped a feature allowing users to perform geospatial search and access realtime nearby transit information. This feature has been highlighted in \href{https://www.wired.com/story/lyft-public-transit-app-zimmer-santa-monica/}{Wired} and \href{https://www.bloomberg.com/news/articles/2019-07-19/lyft-is-adding-new-york-subway-info-to-app-even-as-it-fights-with-the-city}{Bloomberg}. \hfill\emph{Python}
  \end{list2}

\item \begin{tabular1bold} Teachers Pay Teachers, Software Engineer; New York, NY & Sep 2016 -- Feb 2018 \end{tabular1bold}

  \begin{list2}
  \item E-commerce startup which allows teachers to buy and sell teaching materials

  \item Designed and developed a distributed asynchronous task processor service to replace an unreliable legacy system. Contributed significant time to improving an open-source system (\href{https://github.com/edgurgel/verk/commits?author=keyan}{\emph{github.com/edgurgel/verk}}) to power this service.

  \item Guided an organizational migration to Kubernetes by creating a protocol for launching new services, a continuous integration and deployment pipeline for containers, and an internal web application to centralize the deployment process. Some aspects of this work are discussed over two blog posts: \href{https://keyanp.com/tonic.html}{\emph{keyanp.com/tonic.html}}, \href{https://keyanp.com/jenkins.html}{\emph{keyanp.com/jenkins.html}} \hfill\emph{Elixir}

  \item Automated infrastructure management \hfill\emph{Chef, Terraform, AWS}
  \end{list2}

\item \begin{tabular1bold} Venmo (subsidiary of PayPal), Software Engineer; New York, NY & May 2015 -- Aug 2016 \end{tabular1bold}

  \begin{list2}
  \item Worked on the Platform team developing a backend serving millions of users \hfill \emph{Python/Django}
  \item Wore many hats on the small team with projects spanning system reliability and scalability, developer tools and environments, API development, data migrations, test suite refactoring, data security, and automated infrastructure management.
  \end{list2}

\item \begin{tabular1bold} Recurse Center, Recurser; New York, NY &  Jan 2015 -- Mar 2015 \end{tabular1bold}

  \begin{list2}
  \item A self described ``retreat for programmers'' \href{https://recurse.com}{(\emph{recurse.com})} where I talked about, read about, and programmed computers. Learned how to approach new areas and what questions to ask. Also played a lot of chess.
  \end{list2}

\item \begin{tabular1bold} University of Vermont Medical Center, Research Assistant; Burlington, VT & May 2013 -- Dec 2014 \end{tabular1bold}

  \begin{list2}
  \item Studied genetic regulation of sporulation in the bacterium \emph{Clostridium difficile}. Mentored undergraduates and rotating gradudate students.
  \end{list2}

\end{list1}


%
%-------------------------------------------------------------------------------
%

\section*{Education}

\begin{list1}
  \item\begin{tabular1bold}Non-Degree Graduate Coursework, University of Washington; Seattle, WA & 2020\end{tabular1bold}
  Incentives in Computer Science [Game Theory/Economics]

  \item\begin{tabular1bold}Non-Degree Coursework, University of Vermont; Burlington, VT & 2014\end{tabular1bold}
  Data Structures and Algorithms, Linear Algebra, Discrete Mathematics, Computer Organization

  \item\begin{tabular1bold}B.Sc. Microbiology, University of Vermont; Burlington, VT & 2013\end{tabular1bold}
\end{list1}

%
%-------------------------------------------------------------------------------
%

\section*{Certifications}
\begin{list1}
\item\begin{tabular1bold}Deep Learning Specialization, deeplearning.ai [\href{https://www.coursera.org/account/accomplishments/specialization/MEYXXF6FHDAE}{Coursera}] & 2019\end{tabular1bold}
\end{list1}

%
%-------------------------------------------------------------------------------
%

\section*{Patents}

% \begin{adjustwidth}{1.5em}{1.5em}
\begin{list1}
\item U.S. Patent 16/836,141 “Multi-Modal Route Generation System” (provisional application filed 31 March 2020).
\end{list1}

%
%-------------------------------------------------------------------------------
%

\section*{Publications}

% \begin{adjustwidth}{1.5em}{1.5em}
\begin{list1}
\item \textbf{Pishdadian, K.}, Fimlaid, K.A., \& Shen, A., (2015) SpoIIID-mediated regulation of $\sigma$\textsuperscript{K} function during \emph{Clostridium difficile} sporulation. \textbf{\emph{Molecular Microbiology}}, 95:189--208.
\item Shen, A., Fimlaid, K.A., \& \textbf{Pishdadian, K.}, (2016) Inducing and Quantifying \emph{Clostridium difficile} Spore Formation. \textbf{\emph{Methods in Molecular Biology}}, 1476:129-42.
\item Ribis, J.W., Ravichandran, P., Putnam, E.E., \textbf{Pishdadian, K.}, \& Shen, A. (2017) The Conserved Spore Coat Protein SpoVM Is Largely Dispensable in \emph{Clostridium difficile} Spore Formation. \textbf{\emph{mSphere}}, e00315-17.
\end{list1}
% \end{adjustwidth}

%
%-------------------------------------------------------------------------------
%

\end{document}
